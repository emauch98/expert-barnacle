%%
%% Camera-ready submissions do not need line numbers, and
%% should have this option removed.
%%

\documentclass[fleqn,11pt,lineno]{manuscript}\usepackage[]{graphicx}\usepackage[]{xcolor}
% maxwidth is the original width if it is less than linewidth
% otherwise use linewidth (to make sure the graphics do not exceed the margin)
\makeatletter
\def\maxwidth{ %
  \ifdim\Gin@nat@width>\linewidth
    \linewidth
  \else
    \Gin@nat@width
  \fi
}
\makeatother

\definecolor{fgcolor}{rgb}{0.345, 0.345, 0.345}
\newcommand{\hlnum}[1]{\textcolor[rgb]{0.686,0.059,0.569}{#1}}%
\newcommand{\hlsng}[1]{\textcolor[rgb]{0.192,0.494,0.8}{#1}}%
\newcommand{\hlcom}[1]{\textcolor[rgb]{0.678,0.584,0.686}{\textit{#1}}}%
\newcommand{\hlopt}[1]{\textcolor[rgb]{0,0,0}{#1}}%
\newcommand{\hldef}[1]{\textcolor[rgb]{0.345,0.345,0.345}{#1}}%
\newcommand{\hlkwa}[1]{\textcolor[rgb]{0.161,0.373,0.58}{\textbf{#1}}}%
\newcommand{\hlkwb}[1]{\textcolor[rgb]{0.69,0.353,0.396}{#1}}%
\newcommand{\hlkwc}[1]{\textcolor[rgb]{0.333,0.667,0.333}{#1}}%
\newcommand{\hlkwd}[1]{\textcolor[rgb]{0.737,0.353,0.396}{\textbf{#1}}}%
\let\hlipl\hlkwb

\usepackage{framed}
\makeatletter
\newenvironment{kframe}{%
 \def\at@end@of@kframe{}%
 \ifinner\ifhmode%
  \def\at@end@of@kframe{\end{minipage}}%
  \begin{minipage}{\columnwidth}%
 \fi\fi%
 \def\FrameCommand##1{\hskip\@totalleftmargin \hskip-\fboxsep
 \colorbox{shadecolor}{##1}\hskip-\fboxsep
     % There is no \\@totalrightmargin, so:
     \hskip-\linewidth \hskip-\@totalleftmargin \hskip\columnwidth}%
 \MakeFramed {\advance\hsize-\width
   \@totalleftmargin\z@ \linewidth\hsize
   \@setminipage}}%
 {\par\unskip\endMakeFramed%
 \at@end@of@kframe}
\makeatother

\definecolor{shadecolor}{rgb}{.97, .97, .97}
\definecolor{messagecolor}{rgb}{0, 0, 0}
\definecolor{warningcolor}{rgb}{1, 0, 1}
\definecolor{errorcolor}{rgb}{1, 0, 0}
\newenvironment{knitrout}{}{} % an empty environment to be redefined in TeX

\usepackage{alltt}
%%\usepackage{setspace}
%%\doublespacing
\usepackage{soul}

\newcommand{\beginsupplement}{%
        \setcounter{table}{0}
        \renewcommand{\thetable}{S\arabic{table}}%
        \setcounter{figure}{0}
        \renewcommand{\thefigure}{S\arabic{figure}}%
     }

\title{Template for manuscript}

\author[1]{Author One}
\author[1]{Author Two}
\author[2,3]{Author Three}
\author[1]{Author Four}
\affil[1]{Author one affiliation}
\affil[2]{Author two affiliation}
\affil[3]{Author three affiliation}

\corrauthor[1]{Author Four}{email@address}

\keywords{Keyword1; Keyword2; Keyword3}

\begin{abstract}
This is the abstract.
\end{abstract}

%%%%%%%%%%%%%%%%%%%%%%%%%%%%%%%%%%%%%%%%%%
% R specifications

%%%%%%%%%%%%%%%%%%%%%%%%%%%%%%%%%%%%%%%%%%

\IfFileExists{upquote.sty}{\usepackage{upquote}}{}
\begin{document}

\flushbottom
\maketitle
\thispagestyle{empty}

\section*{Introduction}
...

%%%%%%%%%%%%%%%%%%%%%%%%%%%%%%%%%%%%%%%%%%
\section*{Materials and Methods}
...

%%%%%%%%%%%%%%%%%%%%%%%%%%%%%%%%%%%%%%%%%%
\section*{Results}
...
 
%%%%%%%%%%%%%%%%%%%%%%%%%%%%%%%%%%%%%%%%%%
\section*{Discussion}
...

%%%%%%%%%%%%%%%%%%%%%%%%%%%%%%%%%%%%%%%%%%
\section*{Conclusions}
...

%%%%%%%%%%%%%%%%%%%%%%%%%%%%%%%%%%%%%%%%%%
\section*{Author contributions}
...

\section*{Institutional review}
...

\section*{Data availability} 
...

\section*{Funding}
...

\section*{Acknowledgments}
...

\section*{Conflicts of interest}
...

%%%%%%%%%%%%%%%%%%%%%%%%%%%%%%%%%%%%%%%%%%

\bibliography{main}

%%%%%%%%%%%%%%%%%%%%%%%%%%%%%%%%%%%%%%%%%%

\clearpage

\beginsupplement
\renewcommand\figurename{Supplementary Figure}
\renewcommand\tablename{Supplementary Table}

\section*{Appendix}

\subsection{Study design and setup} \label{subsec:design}

In 2019, 12 glacial lakes ($N =$ 12) were chosen for an experiment near Narsarsuaq,
Greenland (61.1567\textdegree{}N, -45.4254\textdegree{}E).
These glacial lakes vary in area (0.21 to 1.82 hectares) and maximum
depth (2 to 8 m), but are all clustered within a few
kilometers of each other. All lakes were fishless at the beginning of the experiment.
Six lakes were subsequently introduced with three-spined sticklebacks
(\textit{Gasterosteus aculeatus}) from nearby lakes.
Lake B1P1, B2P2, and B3P3 were introduced with \textit{Gasterosteus aculeatus} from a single
population (lake L26, 61.253333\textdegree{}N, -45.529141\textdegree{}E),
while lake B2P3, B3P1, and B3P2 were introduced with \textit{Gasterosteus aculeatus} from
two populations (lake L26, 61.253333\textdegree{}N, -45.529141\textdegree{}E
and lake ERL33, 61.118369\textdegree{}N, -45.580845\textdegree{}E).
The remaining six lakes B1P4, B2P4, B3P0, ERL85, ERL122, and ERL152 were
used as fishless control. For the purpose of this study, the origin
of the introduced \textit{Gasterosteus aculeatus} is of minor importance,
as they all originate from the same area (Supplementary Table~\ref{tab:lakes}).

In 2021, 2022, and 2023, all 12 lakes were monitored over several days.
For that purpose, EXO2 multiparameter sondes were installed
(YSI, Yellow Springs, OH, USA), tracking ecosystem parameters with high
frequency (2-minute intervals in 2021 and 2022, 5-minute intervals in
2023 with the exception of ERL122, which was monitored in 15-minute intervals).
For the purpose of this study, only dissolved oxygen and temperature measurements
yielded from these sondes are relevant. The sensors were situated at a water depth
of approximately 1-1.5 m in each lake. All optical sensors were
wiped clean before every measurement with a built-in wiper.
The monitoring period was 16 September-24 September in 2021, 22 June-3 July in 2022,
and 22 June-17 July in 2023.

% latex table generated in R 4.4.1 by xtable 1.8-4 package
% Thu Dec  5 16:44:34 2024
\begin{table}[ht]
\centering
\caption{Lakes included in the experiment, along with treatment
                and general characteristics.} 
\label{tab:lakes}
\begin{tabular}{lllllll}
  \toprule
Lake & Treatment & Latitude (°N) & Longitude (°E) & Altitude (m) & Area (hectare) & Maximum Depth (m) \\ 
  \midrule
B1P1 & Fish & 61.15338 & -45.57081 & 272 & 0.21 & 4.00 \\ 
  B2P2 & Fish & 61.12299 & -45.55988 & 255 & 0.50 & 3.00 \\ 
  B3P3 & Fish & 61.13385 & -45.57556 & 258 & 0.30 & 5.00 \\ 
  B2P3 & Fish & 61.12275 & -45.55696 & 261 & 0.41 & 4.25 \\ 
  B3P1 & Fish & 61.13130 & -45.51195 & 180 & 0.40 & 2.00 \\ 
  B3P2 & Fish & 61.12788 & -45.51031 & 201 & 0.53 & 4.50 \\ 
  B1P4 & No Fish & 61.16552 & -45.56801 & 304 & 0.44 & 2.20 \\ 
  B2P4 & No Fish & 61.12192 & -45.55497 & 261 & 0.51 & 7.00 \\ 
  B3P0 & No Fish & 61.13210 & -45.51416 & 177 & 0.81 & 4.00 \\ 
  ERL85 & No Fish & 61.14171 & -45.59328 & 120 & 1.82 & 8.00 \\ 
  ERL122 & No Fish & 61.14182 & -45.53623 & 111 & 1.10 & 5.00 \\ 
  ERL152 & No Fish & 61.14646 & -45.59235 & 156 & 0.73 & 4.50 \\ 
   \bottomrule
\end{tabular}
\end{table}


\subsection{Data sources} \label{subsec:source}

Dissolved oxygen and water temperature measurements were yielded from EXO2 multiparameter
sondes, as described in section~\ref{subsec:design}. For the purpose
of estimating ecosystem metabolism, wind and irradiation data were yielded
from the QAS\_L automated weather station near Narsarsuaq, Greenland \citep{metadata}.

\subsubsection*{Weather data}

\subsection{Data preparation}

\subsection{Statistical analysis}

\subsection{Implementation}

\end{document}

